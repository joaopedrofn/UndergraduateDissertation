% %--------------------------------------------------------------------------------------
% % Este arquivo contém a sua conclusão
% %--------------------------------------------------------------------------------------
\chapter{Conclusão}\label{sec:conclusion}

Neste capítulo, primeiramente, é feita uma discussão sobre os resultados obtidos
a fim de destacar as contribuições para o futuro do projeto FutVasf2D e o desenvolvimento
de um módulo de defesa para um time de futebol de robôs da liga de simulação 2D.

É possível observar que os gráficos dos resultados dos Modelos
\ref{model:2cycles} e \ref{model:2cycles} apresentam sinais de evolução na
inteligência dos agentes quanto a tentativa de evitar gols. A análise destes
gráficos, apresentados na Figura \ref{img:graph}, mostra um comportamento
semelhante a ruídos, o que é um comportamento espererado devido a, primeiro, a
característica da simulação que cria ruídos nos sensores dos agentes para que o
mesmos se assemelhem aos sensores falhos dos seres humanos e a característica
aleatória da criação de situações do sistema de treinamento.

Apesar desta observação, como demonstrado na mesma figura e nas Equações
\ref{eq:1cycle} e \ref{eq:2cycles}, que se tratam de uma regressão lineas dos
resultados encontrados, é possível encontrar um comportamento de melhoria nos
modelos de mundo em questão. Se baseando em tais equações, é possível extrapolar
um possível valor para o número de blocos a ser executados para que possa ser
superado a média geral do time original, para o Modelo \ref{model:2cycles}, este
número seria de 5265 blocos (mais de 15.000.000 ciclos), já para o Modelo
\ref{model:1cycle} este número seria de 4265 blocos (mais de 12.000.000 ciclos).

Diante dos resultados obtidos através das seções de treinamento, é possível
tomar notas sobre o que pode levar a evolução do projeto e o desenvolvimento de
um módulo de defesa ideal.

Levando em conta todos os modelos de mundo testados e os valores obtidos, é possível
comparar e observar que a tentativa de avaliar a ação tomada após alguns ciclos
de sua execução não apresentou os frutos esperados, pelo contrário, os dois
modelos em destaque foram aqueles com menor número de ações na fila. Também é
possível observar que a primeira abordagem de estados se mostrou mais
eficiente que a segunda.

O resultado deste trabalho serve para apontar novos caminhos a serem tomados
para melhoria de mecanismos de defesa, como o refino dos modelos de mundo aqui indicados
ou a especialização em subtarefas da defesa. Vale a pena focar nos modelos com
melhores resultados deste trabalhos a fim de reproduzi-los e melhora-los.

Uma abordagem interessante a ser tomada para o desenvolvimento de novos modelos
de mundo pode ser a análise do mecanismo já utilizado pelo time original a fim
utilizar o conhecimento já utilizado com o sistema de condicionais para o
melhoramento no campo de aprendizado por reforço.

Também se faz interessante a realização de pesquisas de aplicação do
\textit{Q-Learning} em outros tipos de mecanismos, como no drible ou passe de
bola.

Durante as seções de treinamento foi percebida a necessidade do trabalho no
mecanismo do goleiro, que funciona de forma separada do mecanismo de defesa dos
jogadores comuns. A melhoria do goleiro pode inclusive afetar a forma como os
demais jogadores aprendem, já que com um goleiro mais forte os jogadores teriam
maior liberdade de movimento.

Uma outra possibilidade de pesquisa é a de aplicação de outras técnicas de
aprendizagem de máquina para o mecanismo aqui pesquisado como fim de comparação.

\section{Dificuldades Encontradas}

Durante a execução deste trabalho foram encontrados alguns desafios que
prejudicaram o andamento do mesmo, entre estes pontos estão:

\begin{itemize}
    \item \textbf{Falta de documentação detalhada do time base}, é verdade que o
    time base escolhido (\textit{WrightEagleBASE}) oferece a
    documentação mais completa dentre os principais, mas ainda assim, não é
    fornecido detalhes das funções e classes disponíveis no \textit{framework}
    do mesmo. Isto causa problemas como o desenvolvimento de recursos já
    disponíveis e por vezes menos eficazes;
    \item \textbf{Falta de consistência nos recursos do \textit{framework} do
    time base}, ainda relacionado ao problema anterior, a falta de consistência
    dos recursos causava a utilização de recursos que, embora perfeitamente
    funcionais em certas áreas, não funcionam no local e momento necessários;
    \item \textbf{Concorrência no acesso à \textit{Q-Table}}, este ponto já foi
    abordado na seção \ref{implementacao} e se trata no problema ao acesso para
    leitura e escrita do arquivo relativo à \textit{Q-Table} que fazia com que
    cada agente sobreescrevesse a tabela com seus próprios valores perdendo os
    valores anteriores. Este problema foi contornado através da criação de
    vários arquivos, um para cada agente;
    \item \textbf{Falha na compatibilidade com HFO}. Uma verdade na execução deste trabalho é de que,
    pela sua natureza, foi necessário a execução de treinamentos extensos para
    vários modelos de mundo diferentes. Isto por si só não consistiria em problema, se não
    fosse uma pequena falha na projeção do time base que impedia a plena
    compatibilidade com o HFO. \\
    A falha em questão corresponde a um problema de memória e acesso à ponteiros que
    causa a interrupção dos clientes durante as seções de treinamento, a correção
    desta falha não foi possível por conta da complexidade do código-fonte do time
    base desenvolvido por anos por componentes da equipe \textit{WrightEagle}. Isto
    somado à necessidade de iniciar as seções manualmente, o que impossibilitava a
    escrita de um \textit{script} que reconhecesse a queda dos clientes e o
    executassem novamente, impediu a execução de seções de treinamento longas e
    initerruptas;
    \item \textbf{Quantidade de recursos exigidos}. Outro prolema relacionado à execução dos treinamentos é quantidade de recursos
    utilizadas pelo treinamento, já que o HFO utiliza o máximo que consegue dos
    recursos para executar ciclos rápidos agilizando o treinamento. O consumo em
    questão impossibilitou a execução de treinamento de mais modelos de mundo
    simultaneamente.
\end{itemize}

Tudo isso somado a quantidade de variáveis que precisavam ser testadas para que
se pudesse obter uma otimização do modelo e da implementação do mesmo impediu
exploração maior nas possibilidades levantadas.