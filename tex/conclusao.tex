% %--------------------------------------------------------------------------------------
% % Este arquivo contém a sua conclusão
% %--------------------------------------------------------------------------------------
\chapter{Conclusão}\label{sec:conclusion}

Os resultados encontrados através da execução dos experimentos são longe de
ideais, visto que não foi alcançada uma taxa de gols que conseguisse ser menor
que a do time original, porém, é possível tomar notas sobre o processo que pode
ter levado a tais resultados e inferir possíveis ações que podem levar ao
resultado desejado.

Uma verdade na execução deste trabalho é de que,
pela sua natureza, foi necessário a execução de treinamentos extensos para
vários modelos de mundo diferentes. Isto por si só não consistiria em problema, se não
fosse uma pequena falha na projeção do time base que impedia a plena
compatibilidade com o HFO.

A falha em questão corresponde a um problema de memória e acesso à ponteiros que
causa a interrupção dos clientes durante as seções de treinamento, a correção
desta falha não foi possível por conta da complexidade do código-fonte do time
base desenvolvido por anos por componentes da equipe \textit{WrightEagle}. Isto
somado à necessidade de iniciar as seções manualmente, o que impossibilitava a
escrita de um \textit{script} que reconhecesse a queda dos clientes e o
executassem novamente, impediu a execução de seções de treinamento longas e
initerruptas.

Outro prolema relacionado à execução dos treinamentos é quantidade de recursos
utilizadas pelo treinamento, já que o HFO utiliza o máximo que consegue dos
recursos para executar ciclos rápidos agilizando o treinamento. O consumo em
questão impossibilitou a execução de treinamento de mais modelos de mundo
simultaneamente.

Tudo isso somado a quantidade de variáveis que precisavam ser testadas para que
se pudesse obter uma otimização do modelo e da implementação do mesmo impediu o
avanço do projeto em tempo ábil.

Mas ainda assim, é possível observar que os gráficos dos resultados dos Modelos
\ref{model:2cycles} e \ref{model:2cycles} apresentam sinais de evolução na
inteligência dos agentes quanto a tentativa de evitar gols.

Levando em conta todos os modelos de mundo testados e os valores obtidos, é possível
comparar e observar que a tentativa de avaliar a ação tomada após alguns ciclos
de sua execução não apresentou os frutos esperados, pelo contrários, os dois
modelos em destaque foram aqueles com menor número de ações na fila. Também é
possível observar que a abordagem de estados por variáveis se mostrou mais
eficiente que aquela por papéis situacionais.

Estes resultados, embora não o que se esperava ao se iniciar o projeto, são útei
para execução de novas pesquisas na área, principalmente dando continuidade a
esta pesquisa e ao projeto FutVasf2D.

\section{Trabalhos Futuros}\label{sec:future}

O resultado deste trabalho serve para apontar novos caminhos a serem tomados
para melhoria de mecanismos de defesa, como o refino dos modelos de mundo aqui indicados
ou a especialização em subtarefas da defesa. Vale a pena focar nos modelos com
melhores resultados deste trabalhos a fim de reproduzi-los e melhora-los.

Também se faz interessante a realização de pesquisas de aplicação do
\textit{Q-Learning} em outros tipos de mecanismos, como no drible ou passe de
bola.

Uma outra possibilidade de pesquisa é a de aplicação de outras técnicas de
aprendizagem de máquina para o mecanismo aqui pesquisado como fim de comparação.