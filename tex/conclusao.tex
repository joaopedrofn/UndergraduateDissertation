% %--------------------------------------------------------------------------------------
% % Este arquivo contém a sua conclusão
% %--------------------------------------------------------------------------------------
\chapter{Considerações Finais}\label{sec:conclusion}

O resultado deste trabalho serve para apontar novos caminhos a serem tomados
para melhoria de mecanismos de defesa, como o refino dos modelos de mundo aqui indicados
ou a especialização em subtarefas da defesa. Vale a pena focar nos modelos com
melhores resultados deste trabalho a fim de reproduzi-los e melhorá-los.

Uma abordagem interessante a ser tomada para o desenvolvimento de novos modelos
de mundo pode ser a análise do mecanismo já utilizado pelo time original a fim
utilizar o conhecimento já utilizado com o sistema de condicionais para o
melhoramento no campo de aprendizado por reforço.

Também se faz interessante a realização de pesquisas de aplicação do
\textit{Q-Learning} em outros tipos de mecanismos, como no drible ou passe de
bola.

Durante a execução do treinamento foi percebida a necessidade do trabalho no
mecanismo do goleiro, que funciona de forma separada do mecanismo de defesa dos
jogadores comuns. A melhoria do goleiro pode inclusive afetar a forma como os
demais jogadores aprendem, já que com um goleiro mais forte os jogadores teriam
maior liberdade de movimento.

Uma outra possibilidade de pesquisa é a de aplicação de outras técnicas de
aprendizagem de máquina para o mecanismo aqui pesquisado com o objetivo de comparação.

\section{Dificuldades Encontradas}

Durante a execução deste trabalho foram encontrados alguns desafios que
prejudicaram o andamento do mesmo, entre estes pontos estão:

\begin{itemize}
    \item \textbf{Falta de documentação detalhada do time base}, é verdade que o
    time base escolhido (\textit{WrightEagleBASE}) oferece a
    documentação mais completa dentre os principais, mas ainda assim, não são
    fornecidos detalhes das funções e classes disponíveis no \textit{framework}
    do mesmo. Isto causa problemas como o desenvolvimento de recursos já
    disponíveis e por vezes menos eficazes;
    \item \textbf{Falta de consistência nos recursos do \textit{framework} do
    time base}, ainda relacionado ao problema anterior, foi percebida a falta de
    consistência em alguns recursos oferecidos pelo time base, um exemplo
    simples é a obtenção de dos valores do agente, que por vezes deveria ser
    feito utilizando a classe \textit{Agent} enquanto em outras este valor fica
    inacessível a partir desta e passa a ser acessível apenas através do classes
    como \textit{WorldState};
    \item \textbf{Concorrência no acesso à \textit{Q-Table}}, este ponto já foi
    abordado na seção \ref{implementacao} e se trata do problema de acesso para
    leitura e escrita do arquivo relativo à \textit{Q-Table} que fazia com que
    cada agente sobrescrevesse a tabela com seus próprios valores perdendo os
    valores anteriores. Este problema foi contornado através da criação de
    vários arquivos, um para cada agente;
    \item \textbf{Falha na compatibilidade com HFO}. Pela natureza deste trabalho, foi necessário a execução de treinamentos extensos para
    vários modelos de mundo diferentes. Isto por si só não consistiria em problema, se não
    fosse uma pequena falha na projeção do time base que impedia a plena
    compatibilidade com o HFO. \\
    A falha em questão corresponde a um problema de memória e acesso a ponteiros que
    causa a interrupção dos clientes durante as sessões de treinamento, a correção
    desta falha não foi possível por conta da complexidade do código-fonte do time
    base desenvolvido por anos por componentes da equipe \textit{WrightEagle}. Isto
    somado à necessidade de iniciar as sessões manualmente impossibilitava a
    escrita de um \textit{script} que reconhecesse a queda dos clientes e o
    executassem novamente, impedindo a execução de sessões de treinamento longas e
    initerruptas;
    \item \textbf{Quantidade de recursos exigidos}. Outro prolema relacionado à execução dos treinamentos é quantidade de recursos
    utilizadas pelo treinamento, já que o HFO utiliza o máximo que consegue dos
    recursos para executar ciclos rápidos agilizando o treinamento. O consumo em
    questão impossibilitou a execução de treinamento de mais modelos de mundo
    simultaneamente.
\end{itemize}

Tudo isso somado a quantidade de variáveis que precisavam ser testadas para que
se pudesse obter uma otimização do modelo e da implementação do mesmo impediu
exploração maior nas possibilidades levantadas. Logo, a forma de tornar o time
gerado a partir das modelagens descritas neste trabalho competitivo, é tentar
otimizar a utilização do tempo na continuação do projeto FutVasf2D.

Este trabalho surgiu como parte do já mencionado projeto FutVasf2D e o mesmo vem
mostrando frutos. Recentemente foi aprovado um artigo para apresentação na
Conferência Brasileira sobre Sistemas Inteligentes (BRACIS, do inglês
\textit{Brazilian Conference on Intelligent Systems}) de 2019 e um Artigo de Descrição
de Time (TDP, do inglês \textit{Team Description Paper}) foi aprovado para a
Competição Latino Americana de Robótica (LARC, do inglês
\textit{Latin American Robotics Competition})/Competição Brasileira de Robótica (CBR)
também de 2019.